% This is sigproc-sp.tex -FILE FOR V2.6SP OF ACM_PROC_ARTICLE-SP.CLS
% OCTOBER 2002
%
% It is an example file showing how to use the 'acm_proc_article-sp.cls' V2.6SP
% LaTeX2e document class file for Conference Proceedings submissions.
% ----------------------------------------------------------------------------------------------------------------
% This .tex file (and associated .cls V2.6SP) *DOES NOT* produce:
%       1) The Permission Statement
%       2) The Conference (location) Info information
%       3) The Copyright Line with ACM data
%       4) Page numbering
%
%  However, both the CopyrightYear (default to 2002) and the ACM Copyright Data
% (default to X-XXXXX-XX-X/XX/XX) can still be over-ridden by whatever the author
% inserts into the source .tex file.
% e.g.
% \CopyrightYear{2003} will cause 2003 to appear in the copyright line.
% \crdata{0-12345-67-8/90/12} will cause 0-12345-67-8/90/12 to appear in the copyright line.
%
% ---------------------------------------------------------------------------------------------------------------
% It is an example which *does* use the .bib file (from which the .bbl file
% is produced).
% REMEMBER HOWEVER: After having produced the .bbl file,
% and prior to final submission,
% you need to 'insert'  your .bbl file into your source .tex file so as to provide
% ONE 'self-contained' source file.
%
% Questions regarding SIGS should be sent to
% Adrienne Griscti ---> griscti@acm.org
%
% Questions/suggestions regarding the guidelines, .tex and .cls files, etc. to
% Gerald Murray ---> murray@acm.org 
%
% For tracking purposes - this is V2.6SP - OCTOBER 2002


\documentclass[12pt]{article}

\setlength{\oddsidemargin}{0in}
\setlength{\evensidemargin}{0in}
\setlength{\topmargin}{0in}
\setlength{\headheight}{0in}
\setlength{\headsep}{0in}
\setlength{\textwidth}{6in}
\setlength{\textheight}{9in}
\setlength{\parindent}{0in} 

\usepackage{graphicx} %For jpg figure inclusion
\usepackage{times} %For typeface
\usepackage{epsfig}
\usepackage{color} %For Comments
%\usepackage[all]{xy}
\usepackage{float}
%\usepackage{subfigure} 
\usepackage{hyperref}
\usepackage{url}
\usepackage{parskip}
\usepackage{upquote}

%% Elena's favorite green (thanks, Fernando!)
\definecolor{ForestGreen}{RGB}{34,139,34}
% Uncomment this if you want to show work-in-progress comments
%\newcommand{\comment}[1]{{\bf \tt  {#1}}}
% Uncomment this if you don't want to show comments
\newcommand{\comment}[1]{}
\newcommand{\emcomment}[1]{\textcolor{ForestGreen}{\comment{Elena: {#1}}}}
\newcommand{\todo}[1]{\textcolor{blue}{\comment{To Do: {#1}}}}

\newcommand{\pscomment}[1]{\textcolor{red}{\comment{Paul: {#1}}}}
\newcommand{\mmcomment}[1]{\textcolor{magenta}{\comment{Max: {#1}}}}
\begin{document}
\pagestyle{plain}
%
% --- Author Metadata here ---
%\conferenceinfo{WOODSTOCK}{'97 El Paso, Texas USA}
%\setpagenumber{50}
%\CopyrightYear{2002} % Allows default copyright year (2002) to be
%over-ridden - IF NEED BE. 
%\crdata{0-12345-67-8/90/01}  % Allows default copyright data
%(X-XXXXX-XX-X/XX/XX) to be over-ridden. 
% --- End of Author Metadata ---

\title{Clojure User Scenarios}
%\subtitle{[Extended Abstract \comment{DO WE NEED THIS?}]
%\titlenote{}}
%
% You need the command \numberofauthors to handle the "boxing"
% and alignment of the authors under the title, and to add
% a section for authors number 4 through n.
%
% Up to the first three authors are aligned under the title;
% use the \alignauthor commands below to handle those names
% and affiliations. Add names, affiliations, addresses for
% additional authors as the argument to \additionalauthors;
% these will be set for you without further effort on your
% part as the last section in the body of your article BEFORE
% References or any Appendices.

\author{
Paul Schliep and Aaron Lemmon \\
Computer Science Discipline \\
University of Minnesota Morris\\
Morris, MN 56267\\
schli202@umn.edu, lemmo031@umn.edu
}

\date{}

\maketitle
\thispagestyle{empty}




\newpage
\setcounter{page}{1}

\section{Day 1}
Suzie starts her first “Hello World” program on the first day of class in a repl from our project. Here are all her code attempts:

\begin{verbatim}
	(print Hello World)
\end{verbatim}

{\addtolength{\leftskip}{10mm}
	\subsubsection*{Clojure message:}

	\verb|java.lang.Exception: Unable to resolve symbol: Hello in this context|

	\subsubsection*{Our project message:}
	
	\verb|Some stuff|

	\subsubsection*{Hint/Explanation:}

This breaks because Clojure thinks that Hello is a symbol (identifier) when Suzie 	wanted it to be just plain text. Our program could respond with a suggestion that if the user wanted the phrase to be plain text to surround it with double quotes. It could also suggest to double-check that the user spelled the symbol correctly.

}

\begin{verbatim}
	print("Hello World")
\end{verbatim}

{\addtolength{\leftskip}{10mm}

	\subsubsection*{Clojure message:}
	
	\verb|some stuff|

	\subsubsection*{Our project message:}
	
	\verb|some stuff|
	
	\subsubsection*{Hint/Explanation:}

This breaks because Clojure thinks the first thing after an open paren should be a function. Our program could respond with a suggestion to the user to make sure that the item right after the open paren is a function, otherwise check syntax. 

}

\begin{verbatim}
	(print "Hello World")
\end{verbatim}

{\addtolength{\leftskip}{10mm}

	\subsubsection*{Hint/Explanation:}
	
	This works!
	
}


\section{Day 2}
For the second day, Jaden wants to define a function in a repl from within our project. Here are all his code attempts:

\begin{verbatim}
	(fn squareThis input*input)
\end{verbatim}

{\addtolength{\leftskip}{10mm}

	\subsubsection*{Clojure message:}
	
	\verb|java.lang.RuntimeException: java.lang.UnsupportedOperationException: |
	
	\verb|nth not supported on this type: Symbol|

	\subsubsection*{Our project message:}
	
	\verb|some stuff|
	
	\subsubsection*{Hint/Explanation:}
In this particular case, fn was expecting a vector of parameters and it didn't find one. In the general case, something was looking for a collection and didn't find one.
	
}

\begin{verbatim}
	(fn squareThis [x] (* x x))
	(squareThis 5)
\end{verbatim}

{\addtolength{\leftskip}{10mm}

	\subsubsection*{Clojure message:}
	
	\verb|some stuff|

	\subsubsection*{Our project message:}
	
	\verb|some stuff|
	
	\subsubsection*{Hint/Explanation:}
	
Unable to resolve symbol “squareThis”. This breaks because fn doesn’t make the function last like defn does. fn is only temporary. Our program suggest that the user makes sure he/she defined the symbol correctly.

}

\begin{verbatim}
	(defn squareThis [x] (* x x))
\end{verbatim}

{\addtolength{\leftskip}{10mm}
	
	\subsubsection*{Hint/Explanation:}
	
	This works!
	
}

\section{Day 3}
On the third day of Clojure, Laken is super-excited to start working with lists! She has a list of her favorite bands and she wants to add a new one. Here are her attempts to do this in Clojure:

\begin{verbatim}
	(conj "ACDC" ("Daft Punk" "U2" "ZZ Top"))
\end{verbatim}

{\addtolength{\leftskip}{10mm}

	\subsubsection*{Clojure message:}
	
	\verb|some stuff|

	\subsubsection*{Our project message:}
	
	\verb|some stuff|
	
	\subsubsection*{Hint/Explanation:}
	
This breaks because there is no single quote on the parentheses, so Clojure thinks that the first thing after the open paren is a function, but a string was provided instead. Our program could suggest to the user if they are trying to use lists, to put a single quote before the open paren.
	
}

\begin{verbatim}
	(conj "ACDC" '("Daft Punk" "U2" "ZZ Top"))
\end{verbatim}

{\addtolength{\leftskip}{10mm}

	\subsubsection*{Clojure message:}
	
	\verb|some stuff|

	\subsubsection*{Our project message:}
	
	\verb|some stuff|
	
	\subsubsection*{Hint/Explanation:}
	
This breaks because the arguments conj are in the wrong order. When class cast exceptions like these are thrown our program could suggest double-checking the correct order of arguments.
	
}

\begin{verbatim}
	(conj '("Daft Punk" "U2" "ZZ Top") "ACDC")
\end{verbatim}

{\addtolength{\leftskip}{10mm}
	
	\subsubsection*{Hint/Explanation:}
	
This works!
	
}

\section{Day 4}
On the fourth day of Clojure, Keylan is a little bit nervous to work with Maps. He wants to update a value associated with a key in a map he has created. Here are his attempts:

\begin{verbatim}
	(assoc :a 3 {:a 5, :b 8, :c 9})
\end{verbatim}

{\addtolength{\leftskip}{10mm}

	\subsubsection*{Clojure message:}
	
	\verb|some stuff|

	\subsubsection*{Our project message:}
	
	\verb|some stuff|
	
	\subsubsection*{Hint/Explanation:}
	
This breaks because Clojure is trying to apply assoc to :a where it expects a collection to update. So, our program could suggest to the user to make sure their arguments are in the right order.
	
}

\begin{verbatim}
	(assoc (:a 3) {:a 5, :b 8, :c 9})
\end{verbatim}

{\addtolength{\leftskip}{10mm}

	\subsubsection*{Clojure message:}
	
	\verb|some stuff|

	\subsubsection*{Our project message:}
	
	\verb|some stuff|
	
	\subsubsection*{Hint/Explanation:}
	
This breaks because Clojure detects that the wrong number of arguments were passed to the function. assoc expects an odd number of arguments but only 2 were passed in. Our program can tell the user to double-check the function definition to see how many arguments it expects.
	
}

\begin{verbatim}
	(assoc {:a 3} {:a 5, :b 8, :c 9})
\end{verbatim}

{\addtolength{\leftskip}{10mm}

	\subsubsection*{Clojure message:}
	
	\verb|some stuff|

	\subsubsection*{Our project message:}
	
	\verb|some stuff|
	
	\subsubsection*{Hint/Explanation:}
	
This breaks for the same reason as above

}

\begin{verbatim}
	(assoc {:a 5, :b 8, :c 9} :a 3)
\end{verbatim}

{\addtolength{\leftskip}{10mm}

	\subsubsection*{Clojure message:}
	
	\verb|some stuff|

	\subsubsection*{Our project message:}
	
	\verb|some stuff|
	
	\subsubsection*{Hint/Explanation:}
	
This works!

}

\section{Day 5}
On the fifth day of Clojure, Addison needs to write a function which takes a sequence and returns the maximum value from the sequence without using max or key-max. Here are Addison's attempts:

\begin{verbatim}
	(defn [coll] penultimate (last (drop-last coll)))
\end{verbatim}

{\addtolength{\leftskip}{10mm}

	\subsubsection*{Clojure message:}
	
	\verb|some stuff|

	\subsubsection*{Our project message:}
	
	\verb|some stuff|
	
	\subsubsection*{Hint/Explanation:}
	
This throws an IllegalArgumentException: Don't know how to create ISeq from: clojure.lang.Symbol. In this specific case we could give a hint that the function name and the parameter list are out of order.

}

\begin{verbatim}
	(defn penultimate [coll] (last (drop-last coll)))
\end{verbatim}

{\addtolength{\leftskip}{10mm}

	\subsubsection*{Clojure message:}
	
	\verb|some stuff|

	\subsubsection*{Our project message:}
	
	\verb|some stuff|
	
	\subsubsection*{Hint/Explanation:}
	
This works!

}

% That's all folks!
\end{document}

%%%%%%%%%%%%%%%%%%%%%%%%%%%%%%%%%%%%%%%%%%%%%%%%%%%%%%%%%%%%%%%%
